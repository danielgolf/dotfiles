\documentclass[
    a4paper,
    10pt,
    %fleqn,
    notitlepage,
    parskip=half,
    egregdoesnotlikesansseriftitles
]{scrartcl}
\usepackage[english, ngerman]{babel}
\usepackage[utf8]{inputenc}
\usepackage[T1]{fontenc}
\usepackage{microtype}
\usepackage{hyperref}
\usepackage{amsmath, amssymb, mathtools}
\usepackage[margin=20mm]{geometry}

\usepackage{kpfonts}
%\usepackage[modulo]{lineno}
%\def\linenumberfont{\normalfont\tiny}
%\usepackage{multicol}
%\setlength{\columnsep}{12mm}

%\usepackage[headsepline]{scrlayer-scrpage}
%\ohead{left}
%\chead{mid}
%\ihead{right}

% BibLateX
%\usepackage[backend=biber, url=false, bibencoding=inputenc]{biblatex}
%\usepackage{csquotes}
%\addbibresource{quellen.bib}

%\usepackage{tabularx}
%\usepackage{booktabs}
%\usepackage{diagbox}

%\usepackage{tikz}
%\usepackage{subfig}
%\usepackage{caption}
%\usepackage{graphicx}
%\usepackage[section]{placeins}

%\usepackage{xcolor}
%\usepackage{listings}
\usepackage[linesnumbered,ruled,vlined]{algorithm2e}

%\usepackage{ifthen}
%\newboolean{loesung}                                % Deklaration
%\setboolean{loesung}{false}                         % Zuweisung

%\graphicspath{{imgs/}}
\setcounter{secnumdepth}{0}                         % keine Nummerierung von Überschriften
\numberwithin{equation}{section}                    % Formeln mit Überschrift nummerieren
\newcommand{\emptyline}{\vspace{\baselineskip}}     % erzeugt leerzeile

%\newcommand{\N}{\mathbb{N}}
%\newcommand{\Z}{\mathbb{Z}}
%\newcommand{\R}{\mathbb{R}}
%\newcommand{\C}{\mathbb{C}}
%\newcommand{\Pf}{\mathbb{P}}
%\newcommand{\bw}{\mathbf{w}}
\DeclareMathOperator{\var}{var}
%\DeclareMathOperator*{\argmin}{argmin}

\title{Hausaufgabe}
\subtitle{Modul}
\author{Daniel W.}
\date{\today}

\begin{document}

\maketitle
%\newpage
%\tableofcontents
%\linenumbers

\section{Hausaufgabe 1}

Formel: $ a_1 + a_2 = a_3 $

\newpage

\section{Hausaufgabe 2}

Blabllbalbasfasf Hallo

%\ifthenelse{\boolean{loesung}}{true}{false}

\begin{algorithm}
    \caption{Variance of the OLS Estimator}
    %\DontPrintSemicolon
    \KwIn{$n$ (number of data points); $\sigma^2_\epsilon$ (noise variance);
        $\sigma^2_x$ (data variance); $w$ (true slope)}
    \KwOut{variance of $\hat{w}$}
    \emptyline

    Generate $n$ gaussian data points $X = [x_1, \ldots , x_n]$, $x_i \sim \mathcal{N}(0, 1)$ \\
    \For{$r=1,\dots,10^5$}{
        generate $n$ gaussian noise terms $E = [\epsilon_1, \ldots, \epsilon_N]$,
            $\epsilon_i \sim \mathcal{N}(0, \sigma_\epsilon^2)$ \\
        compute $y=w\cdot X + E$ \\
        compute OLS estimate $\hat{w}[r] = (X X^\top)^{-1} X y^\top$
    }
    \Return{$\var(\hat{w})$}
\end{algorithm}

%\newpage
%\printbibliography

\end{document}
